\clearpage
\section{ЗАКЛЮЧЕНИЕ}
Целью выпускной квалификационной работы является демонстрация технологии процедурного кластерного видозависимого лоддирования и определение её ограничений.

Для достижения этой цели были поставлены следующие задачи:
\begin{enumerate}
    \item изучить механизм работы Nanite;
    \item реализовать упрощённую систему процедурного кластерного видозависимого изменения детализации;
    \item определить проблемы, возникающие при реализации;
    \item определить принципиальные ограничения технологии;
    \item сравнить с ,,монолитной`` детализацией.
\end{enumerate}

Для достижения цели и задач был изучен механизм работы Nanite --- первой известной успешно внедрённой в коммерческий движок реализации такой технологии.
Механизм работы Nanite основывается на преобразовании исходного меша в граф мешлетов, в котором поддерживается следующее свойство: независимо от ракурса оценка искажения мешлета-ребёнка не превышает оценку искажения мешлета-родителя.
Это свойство используется для массового параллелизма в отрисовке мешлетов разной детализации без видимых искажений.

Для лучшего понимания механизма работы Nanite реализована упрощённая версия системы и продемонстрирована корректность работы реализованной упрощённой версии, и, как следствие, технологии в целом.
В ходе реализации выявлены принципиальные ограничения такой технологии и существенные технические проблемы, которые необходимо решать при реализации полной версии.

Определены следующие технические проблемы, которые необходимо подробнее рассмотреть при реализации полной версии технологии:
\begin{itemize}
    \item задача разбиения графа;
    \item ограничение размера мешлетов;
    \item оптимизация структуры мешлетов;
    \item организация параллельного спуска по графу;
    \item организация параллельной децимации.
\end{itemize}

Определены следующие существенные принципиальные ограничения технологии, на основании которых можно определить, к каким частям проекта технология применима и потенциальные выгоды от разработки и интеграции полной версии:
\begin{itemize}
    \item оценка искажения не зависит от направления взгляда;
    \item невозможность анимации меша;
    \item неприменимость к некоторым типам мешей;
    \item необходимость использовать меши сверхвысокого разрешения.
\end{itemize}

Проведено сравнение с тривиальным алгоритмом монолитного лоддирования, это сравнение показало, что полученная реализация технологии процедурного кластерного видозависимого лоддирования пока сущетсвенно проигрывает в производительности тривиальному алгоритму монолитного лоддирования.
Для исправления этого в дальнейших работах предлагается исследовать те оптимизации, предложенные в главе 3, которые не были реализованы в данной работе.

Таким образом, поставленные задачи выполнены и поставленная цель выпускной квалификационной работы достигнута.
