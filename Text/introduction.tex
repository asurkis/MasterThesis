\clearpage
\section{ВВЕДЕНИЕ}
Наиболее распространённый метод изображения любого объекта в трёхмерных компьютерных играх --- отрисовка набора треугольников, проецируемых из виртуального трёхмерного пространства на экран.
Такой набор называется ,,мешем`` --- транслитерация устоявшегося английского термина mesh --- обычно он задаёт приближение поверхности объекта, что хорошо подходит для ситуаций, когда можно считать эту поверхность абсолютно непрозрачной.
Целевые платформы компьютерных игр --- компьютеры и игровые приставки --- оснащены видеокартами --- сопроцессорами, аппаратно поддерживающими быструю отрисовку мешей.

Для большей правдоподобности изображения компьютерные игры стремятся увеличивать разрешение мешей --- приближать поверхность объекта большим количеством треугольников малого размера.

Для компьютерных игр критически важно выводить изображение в реальном времени, т.к. оно является реакцией на пользовательский ввод.
Для отрисовки в реальном времени одного объекта в высоком разрешении обычно достаточно тривиального алгоритма, но обычно на экране отображаются сотни объектов одновременно, и для отрисовки их всех в реальном времени в высоком разрешении вычислительных мощностей современных компьютеров обычно недостаточно.

Классическое решение вышеупомянутой проблемы состоит в отрисовке объектов с разными уровнями детализации.
Уровни детализации, также называемые ,,лодами`` от английской аббревиатуры LOD --- Level of Detail --- это набор мешей одного и того же объекта в разных разрешениях.
Уровень детализации для отрисовки объекта выбирается на основании оценки визуального искажения данного уровня детализации с учётом ракурса.
Например, для объекта, расположенного вплотную к камере, будет отрисован меш в максимальном разрешении, а для объекта далеко от камеры можно использовать низкий уровень детализации.

У уровней детализации есть значительные проблемы: их создание требует ручной доработки, а оценка визуального искажения вычислительно сложна.
Также существуют крупные объекты, такие как статуи, скалы и здания, для которых выбор уровня детализации неоднозначен, например, если игрок рассматривает кирпичную стену здания вблизи, то ему должны быть видны мельчайшие детали вплоть до царапин на кирпичах, но для этого здание целиком нужно рисовать в максимальном разрешении, включая ту часть стены, которая находится от игрока на таком расстоянии, что отдельные кирпичи будут едва различимы --- это требует столько же ресурсов, как если бы были видны мельчайшие детали всей стены целиком.

Было много попыток создать другой метод уменьшения сложности вычислений, наиболее примечательные из тех, которые не разбираются в этой работе --- Sparse Voxel Octrees и Megatexture от компании id~Software.
Megatexture не относится к уменьшению сложности обсчёта самих мешей, обработка же вокселей это альтернатива отрисовке набора треугольников --- вместо них используется описание объёма объекта в приближении элементарными объёмами, обычно кубами очень малого размера, например, со стороной 100~мкм.
Основным недостатком вокселей является существенная потеря детализации по сравнению с мешами при условии сравнимого размера сжатой записи.

В 2020 году компания Epic~Games представила свой подход к уменьшению сложности вычислений при отрисовке высокодетализированных объектов --- технологию Nanite в движке Unreal~Engine~5.
Nanite --- технология процедурного кластерного видозависимого лоддирования, то есть:
\begin{itemize}
    \item Nanite автоматически создаёт уровни детализации, без необходимости дорабатывать их вручную;
    \item Nanite отображает части одного объекта с разными подходящими уровнями детализации;
    \item Nanite делает швы между разными уровнями детализации незаметными;
    \item Nanite использует меш максимально возможного разрешения, для которого хватает разрешения экрана и разрешения исходного меша.
\end{itemize}

Несмотря на заявленные преимущества и то, что исходный код Nanite открыт, подобную технологию на момент написания работы в 2024 году пока не внедрили в свои движки:
\begin{itemize}
    \item Unity
    \item Activision
    \item Ubisoft
    \item Crytek
    \item CD Projekt Red
    \item и т.д.
\end{itemize}
За 4 года --- значительный срок для разработчиков компьютерных игр --- Epic Games остаётся единственной компанией, успешно внедрившей технологию процедурного кластерного .
Следовательно, у технологии есть какие-то существенные ограничения, сравнимые с преимуществами.

Целью выпускной квалификационной работы является определение ограничений технологии процедурного кластерного видозависимого лоддирования.
Задачи выпускной квалификационной работы:
\begin{enumerate}
    \item изучить механизм работы Nanite;
    \item реализовать упрощённую систему процедурного кластерного видозависимого лоддирования
    \begin{itemize}
        \item одна из гипотез --- Nanite очень сложен в реализации, для её проверки предлагается реализовать упрощённую версию;
        \item упрощённая версия позволит также проводить объяснимые сравнения с классической реализацией;
    \end{itemize}
    \item определить проблемы, возникающие при реализации;
    \item определить принципиальные ограничения технологии;
    \item сравнить с ,,монолитной`` детализацией.
\end{enumerate}

Информация о технических и принципиальных ограничениях позволит Saber~Interactive принять решение о реализации полной версии технологии и интеграции в существующий движок.
