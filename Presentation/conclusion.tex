\begin{frame}{Результаты}
    \begin{itemize}
        \item Продемонстрирована работоспособность технологии процедурного кластерного видозависимого лоддирования
        \item Определены технические проблемы, которые необходимо решать при реализации полной версии технологии:
        \begin{itemize}
            \item Разбиение меша на мешлеты
            \item Оптимизация децимации
            \item Оптимизация структуры мешлетов
            \item Оптимизация отрисовки треугольников
            \item Организация параллельного спуска по графу
        \end{itemize}
        \item Определены принципиальные ограничения технологии:
        \begin{itemize}
            \item Оценка искажения не зависит от направления взгляда
            \item Невозможность анимации меша
            \item Невозможность автоматической децимации некоторых мешей
            \item Необходимость использовать меши сверхвысокого разрешения
        \end{itemize}
        \item Сравнение с монолитным лоддированием показало, что для внедрения технологии в реальный проект нужны дальнейшие исследования
    \end{itemize}
\end{frame}
