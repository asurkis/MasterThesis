\begin{frame}{Nanite}
    Общий механизм работы Nanite\footnote{\url{https://advances.realtimerendering.com/s2021/Karis_Nanite_SIGGRAPH_Advances_2021_final.pdf}} состоит из двух этапов:
    \begin{itemize}
        \item Предподсчёт --- преобразование модели
        \item Непосредственно отрисовка
    \end{itemize}
\end{frame}

\begin{frame}{Nanite: предподсчёт}
    \begin{itemize}
        \item Меш разбивается на мешлеты
        \item Строится граф, в котором мешлеты --- вершины
        \item Граф похож на дерево
        \item Исходные мешлеты --- листья
        \item \alert{Свойство графа: родитель менее детализирован, чем ребёнок, с любого ракурса}
    \end{itemize}
\end{frame}

\begin{frame}{Nanite: пример предподсчёта, шаг 1}
    \centering \includesvg{metis-0.svg}
\end{frame}

\begin{frame}{Nanite: пример предподсчёта, шаг 2}
    \centering \includesvg{metis-1.svg}
\end{frame}

\begin{frame}{Nanite: пример предподсчёта, шаг 3}
    \centering \includesvg{metis-2.svg}
\end{frame}

\begin{frame}{Nanite: пример предподсчёта, шаг 4}
    \centering \includesvg{metis-3.svg}
\end{frame}

\begin{frame}{Nanite: отрисовка}
    \begin{itemize}
        \item \alert{Решения об отрисовке мешлета независимы для всех мешлетов}
        \item Массовый параллелизм
        \item Корректность благодаря свойству графа
        \item Треугольники в 1 пиксель обрабатываются отдельно
    \end{itemize}
\end{frame}
